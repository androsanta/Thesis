\documentclass[%
    corpo=12pt,
    twoside,
    oldstyle,
    tipotesi=magistrale,
    greek,
    evenboxes,
]{toptesi}

\usepackage[utf8]{inputenc}  % codifica d'entrata
\usepackage[T1]{fontenc}     % codifica dei font
\usepackage{lmodern}         % scelta dei font
\usepackage{hyperref}
\usepackage{dirtree}
\usepackage{listings}
% \usepackage{packages/tslistings} @todo set style
\usepackage{csquotes}
\usepackage{array}
\usepackage{tabularx}

\lstset{%
    frame=single,
    captionpos=b
}

\hypersetup{%
    pdfpagemode={UseOutlines},
    bookmarksopen,
    pdfstartview={FitH},
    colorlinks,
    linkcolor={blue},
    citecolor={blue},
    urlcolor={blue}
}

\interlinea{1.5}

\begin{document}
    \english
    \errorcontextlines=9

    \begin{ThesisTitlePage}
    \TesiDiLaurea{Master Degree Thesis}

    \ateneo{Politecnico di Torino}

    \titolo{Development, Test and Application of a framework for cloud serverless services}
    % \sottotitolo{}
    %%%%%%% Corso degli studi
    \corsodilaurea{Ingegneria Informatica}

    \TitoloListaCandidati{Candidate}
    % @TODO translate matricola in english
    \candidato{Andrea \textsc{Santu}}[251579]

    %%%%%%% Relatori o supervisori
    \relatore{Dr.~Ing.~Boyang Du}
    \AdvisorName{Thesis Supervisor}

    %%%%%%% Tutore
    \tutoreaziendale{Dott.\ Magistrale Antonio Giordano}
    \NomeTutoreAziendale{Intership Tutor}

    %%%%%%% Seduta dell'esame
    \sedutadilaurea{\textsc{Academic~year} 2020-2021}

    %%%%%%% Logo della sede
    \logosede{source/images/polito}

\end{ThesisTitlePage}
    \begin{abstract}
    The overview of services for the creation of web applications is focusing more
    and more towards a micro services oriented approach, moving away from monolithic
    structures.
    The maximum representation of this is with the serverless paradigm, which since
    2014 has seen an ever greater increase in its use and in its investments by the
    major cloud providers.
    Such a paradigm has found an implementation in the cloud model Function as a
    Service, which uses plain simple functions as its main resources.
    Serverless Framework has emerged as one of the major framework that allows the
    usage of the homonym paradigm in a simple way, and it introduced a level of
    abstraction regarding the underlying structure of the chosen cloud provider.
    Despite the functionalities introduced by Serverless, the developer must take
    charge of various operations concerning indirectly the business logic of the
    application, with the main one being: to structure the code base, to define the
    various resources through the compilation of a configuration file, to define a
    unit testing structure, fundamental once the application complexity increases.
    Furthermore, based on the chosen cloud provider, the developer must find solutions
    to problems such as Cold start, and limitations in resources creation.

    The Restlessness framework was born with the goal of improving the user experience
    of Serverless, providing a standard project and testing structure, a Command Line
    Interface and a local Web Interface through which is possible to completely manage
    the project, and with the further goal of minimizing all operations that do not
    concern directly the application's business logic.
    The framework is provided as an Open Source package, and with the possibility of
    extending its functionalities, through the use of addons, some of which are already
    present, to address common patterns, such as database access or authentication.
    During the framework development has been possible to test it on real applications,
    thus allowing to find and correct critical issues, with the main ones being:
    Cold start handling, use of the non relational database mongodb, and limitations
    on the applications structure proposed at the beginning of the framework
    development.

\end{abstract}

% Il panorama dei servizi per la creazione di applicazioni web si sta indirizzando
% sempre più verso un approcio orientato ai micro servizi, allontanandosi dalla
% struttura denominata monolitica.
% La massima rappresentazione di ciò si ha con il paradigma serverless, il quale a
% partire dal 2014 ha visto un sempre maggiore incremento nel suo utilizzo e
% nell' investimento da parte dei maggiori cloud providers.
% Tale paradigma ha trovato realizzazione nel modello cloud Funcions as a Service,
% il quale utilizza come principale risorsa le singole funzioni.
% Serverless Framework è emerso come uno dei maggiori framework che consentono di
% utilizzare l'omonimo paradigma in modo semplice e introducendo un livello di
% astrazione riguardo la struttura sottostante del cloud provider scelto.
% Nonostante le funzionalità introdotte da Serverless, lo sviluppatore deve farsi
% carico di diverse operazioni che riguardano in maniera indiretta la business logic
% del servizio che si vuole creare, e le principali sono: strutturare la code base,
% definire le varie risorse tramite la compilazione di file di configurazione,
% definire una struttura di unit testing, fondamentale una volta che l'applicazione
% aumenta in dimensioni. Inoltre a seconda del cloud provider scelto si dovranno
% trovare soluzioni per fenomeni di Cold start e limitazioni nella creazione delle
% risorse.

% Il framework Restlessness nasce con l'obiettivo di migliorare l'esperienza d'uso
% di Serverless, fornendo una struttura di progetto e di testing standard, una
% Command Line Interface ed una local Web Interface tramite le quali sia possibile
% gestire interamente il progetto, e con l'ulteriore obiettivo di minimizzare tutte
% quelle operazioni che non riguardano direttamente la business logic dell'applicazione.
% Il framework viene fornito in modalità open source, alla possibilità di estendere
% le sue funzionalità tramite l'utilizzo di addons, alcuni dei quali già presenti
% to address common patterns, come l'accesso al database o l'autenticazione.
% Durante lo sviluppo del framework è stato possibile testare lo stesso su applicazioni
% reali, e questo ha permesso di trovare e correggere alcune criticità, le cui
% principali sono state: gestione del cold start, utilizzo del database non relazionale
% mongodb e limitazioni nella struttura delle applicazioni inizialmente proposta.
    % \include{source/acknowledgements}

    \renewcommand{\baselinestretch}{1.1}\normalsize
    \allcontents
    \renewcommand{\baselinestretch}{1.5}\normalsize

    \begin{mainmatter}
        \begin{chapter}{Cloud services}
    \label{chap:cloud_services}

    In the early days of the web, anyone who wanted to build a web application had
    to buy and maintain the physical hardware required to run a server, which was
    a cumbersome process to undertake, especially for small businesses
    \cite{what_is_sls_cloudflare}.
    Then came a new paradigm for the provisioning of computing infrastructure, named
    Cloud Computing, and defined as:

    \enquote*{%
        Clouds are a large pool of easily usable and accessible virtualized resources
        (such as hardware, development platforms and/or services). These resources
        can be dynamically reconfigured to adjust to a variable load (scale), allowing
        also for an optimum resource utilization. This pool of resources is typically
        exploited by a pay-per-use model in which guarantees are offered by the
        Infrastructure Provider by means of customized SLAs.%
    } \cite{cloud_computing_definition}

    \begin{figure}
        \centering
        \includegraphics[width=10cm]{source/images/what-is-the-cloud.png}
        \caption{Representation of the cloud}
    \end{figure}

    Cloud Computing is possible because of a technology called virtualization, which
    allows the creation of a simulated computer, named virtual machine, that behaves
    as if it were a physical computer with its own hardware. When properly implemented,
    this approach allows having a more efficient use of the physical hardware, as
    each computer is able to run many virtual machines at once.
    Despites the many benefits, using virtual machines still requires manual server
    administration, as each one simulate a full system, including the operating
    system and the underlying kernel.

    The next technological step has been containerization, which gave the possibility
    of packing an application and all its dependencies, such as system libraries
    and system settings into a single entity called Container. With this approach
    a single physical machine, including the kernel, is shared by a multitude of
    containers. The main advantages that containerization offers, with respect to
    virtual machines are \cite{what_is_the_cloud}:
    \begin{itemize}
        \item Portability: once the application is packed into a container it can
            be run on any host supporting that technology.
        \item Control and flexibility.
        \item Faster deploy.
        \item Less server administration.
    \end{itemize}
    With this premises about the cloud and its infrastructure is possible to outline
    the main models that have emerged in the context of cloud computing.

    \section{Cloud computing models}
    Among the various types of cloud computing architectures have emerged three
    main models, which are: Infrastructure as a Service, Platform as a Service and
    Software as a Service. Each model is characterized by an increasing level of
    abstraction regarding the underlying infrastructure.

    \begin{figure}
        \centering
        \includegraphics[height=3cm]{source/images/saas-paas-iaas-cloud-pyramid.png}
        \caption{IaaS, PaaS, SaaS Pyramid}
        \label{fig:cloud_computing_pyramid}
    \end{figure}

    \subsection{Infrastructure as a Service (IaaS)}
    Infrastructure refers to the computers and servers than run code and store data.
    A vendor hosts the infrastructure in data centers, referred to as the cloud,
    while customers access it over the Internet. This eliminates the need for customers
    to own and manage the physical infrastructure, so they can build and host web
    applications, store data or perform any kind of computing with a lot more flexibility.
    An advantage of this approach is scalability, as customers can add new servers
    on demand, every time the business needs to scale up, and the same apply also
    if the resources are not needed anymore. Essentially physical servers purchasing,
    installing, maintenance and updating operations are outsourced to the cloud
    provider, so customers can spend fewer resources on that and focus more on business
    operations, thus leading to a faster time to market. The main drawback of this
    approach is the cost effectiveness, as businesses needs to over-purchase resources
    to handle usage spikes, this leads to wasted resources \cite{iaas}.

    \subsection{Platform as a Service (PaaS)}
    This model simplify web development, from a developer perspective, as they can
    rely on the cloud provider for a series of services, which are vendor dependent.
    However some of them can be defined as core PaaS services, and those are: development
    tools, middleware, operating systems, database management, and infrastructure.
    PaaS can be accessed over any internet connection, so developers can work on
    the application from anywhere in the world and build it completely on the browser.
    This kind of simplification comes at the cost of less control over the development
    environment \cite{paas}. An example of this kind of services is Google's
    \href{https://cloud.google.com/appengine}{App Engine}.

    \smallskip
    Another model has recently been added to the three main cloud computing models,
    named Backend as a Service (Baas). This model stands, with some differences,
    at the same level of PaaS, and it's suited especially for web and mobile backend
    development. As with PaaS, BaaS also makes the underlying server infrastructure
    transparent from the developer point of view, and also provides the latter with
    api and sdk that allow the integration of the required backend functionalities.
    The main functionalities already implemented by BaaS are: database management,
    cloud storage, user authentication, push notifications, remote updating and hosting.
    Thanks to these functionalities there may be a greater focus on frontend or mobile
    development.
    In conclusion BaaS provides more functionalities with respect to the PaaS model,
    while the latter provides more flexibility.

    \subsection{Software as a Service (SaaS)}
    In this model the abstraction from the underlying infrastructure is maximized.
    The vendor makes available a fully built cloud application to customers, through
    a subscription contract, so rather than purchasing the resource once there is
    a periodic fee. The main advantages of this model are: access from anywhere,
    no need for updates or installations, scalability, as it's managed by the SaaS
    provider, cost savings.
    However there are also main disadvantages, that makes this solution not suitable
    in some cases: developers have no control over the vendor software, the business
    may become dependent on the SaaS provider (vendor lock-in), no direct control
    over security, this may be an issue especially for large companies \cite{saas}.

    \begin{figure}
        \centering
        \includegraphics[width=\linewidth]{source/images/saas-paas-iaas-diagram.png}
        \caption{IaaS, PaaS, SaaS diagram}
        \label{fig:cloud_computing_architectures}
    \end{figure}

    \section{Serverless paradigm}
    The downsides of the previously described approaches varies from the control on the
    infrastructure and on the software, to scalability problems, to end with cost
    and resources utilization effectiveness.
    Aiming to solve these problems, the major providers started investing on a new
    cloud computing model, named Function as a Service (FaaS) and based on the
    serverless paradigm.
    Such a paradigm is based on providing backend services on an as-used basis, with
    the cloud provider allowing to develop and deploy small piece of code without
    the developer having to deal with the underlying infrastructure.
    So despite the terminology, serverless does not mean without servers, as they are
    of course still required, but they are transparent to developers, which can focus
    on smaller pieces of code.
    With this model, rather than over purchase the resources, to ensure correct
    functionality in all workload situations, as happens in the IaaS model, the vendor
    charges for the actual usage, as the service is auto-scaling. Thanks to this approach
    consumer costs will be fine grained as shown in \ref{fig:serverless_benefits}.

    \begin{figure}
        \centering
        \includegraphics[width=\linewidth]{source/images/benefits-of-serverless.png}
        \caption{Cost Benefits of Serverless}
        \label{fig:serverless_benefits}
    \end{figure}

    Being the underlying infrastructure transparent for the developer, you get the advantage
    of a simpler software development process, and this advantage characterize also
    the PaaS model. Furthermore, being the service auto-scaling, is possible to obtain
    a virtually unlimited scaling capacity, as it happens in the IaaS model, where the
    limit is the cloud provider availability.

    An implementation of the serverless paradigm is the cloud model named Function
    as a Service (FaaS), which allows developers to write and update pieces of code
    on the fly, typically a single function.
    Such code is then executed in response to an event, usually an api call, but other
    options are possible, so it executed regardless of the events, and this lead to
    the previously described benefit regarding scalability and cost effectiveness.
    Furthermore, through this model turns out to be more efficient to implement web
    applications using the modular approach of the micro services architecture
    (\ref{fig:monolithic_to_microservices}), since the code is organized as a set of
    independent functions from the beginning.

    \begin{figure}
        \centering
        \includegraphics[width=\linewidth]{source/images/monolithic-application-microservice-faas.png}
        \caption{Monolithic to Micro services application}
        \label{fig:monolithic_to_microservices}
    \end{figure}

    So the main advantages of the FaaS model are: improved developer speed, built-in
    scalability and cost efficiency. As each approach, there are also drawbacks, in
    this case developers have less control on the system, and an increased complexity when it
    comes to test the application in a local environment.

    The first cloud provider to move into the FaaS director has been Amazon, with the
    introduction of aws lambda in 2014, followed by microsoft and google, with
    azure function and cloud function respectively in 2016.

    \section{Serverless Framework}
    \label{sec:serverless_framework}
    Shortly after the release of the service Aws lambda functions, has been introduced,
    in 2015, the Serverless framework, with the main objective of making development,
    deploy and troubleshoot serverless applications with the least possible overhead.
    The framework consists of an open source Command Line Interface and a hosted
    dashboard, that combined provide developers with serverless application lifecycle
    management. Serverless supports all runtime provided by Aws, corresponding to
    the most popular programming languages such as: Node.js, Python, Ruby, Java,
    Go, .Net, and others are on development.

    Although the serverless framework, given the number of cloud providers supported,
    aim to be platform agnostic, the following examples will be based on the Aws
    provider and on the Node.js programming language.

    The main work units of the framework, according to the FaaS model, are the functions.
    Each function is responsible for a single job, and although is possible to perform
    multiple tasks using a single function, it's not recommended as stated by the
    design principle Separation of concerns \cite{separation_of_concerns}.
    Each function is executed only when triggered by and Event, which can be of different
    type, such as: http api request, scheduled execution and image or file upload.
    Once the developer has defined the function and the events associated to it,
    the framework take care of creating the necessary resources on the provider platform.

    The framework introduces the concept of Services as unit of organization. Each
    service has one or more functions associated to it and an application can then
    be composed by multiple services. This structure reflects the modular approach
    of the micro services architecture described previously. Finally various applications
    are grouped under an organization (\ref{fig:sls_resource_scheme})

    \begin{figure}
        \centering
        \includegraphics[width=6.5cm]{source/diagrams/serverless_app_service.png}
        \caption{Serverless framework resources scheme}
        \label{fig:sls_resource_scheme}
    \end{figure}

    A service is described by a file, located at the root directory of the project,
    and composed in the format \href{https://yaml.org/}{Yaml} or Json.
    Below is a simple serverless.yml file (listing \ref{lst:simple_sls_yml}), it
    defines the service users, which contains just a function, responsible of creating
    a user. The handler field specify the path to the function code, in this case
    the framework will search for a handler.js file, exporting a usersCreate function,
    as show on listing \ref{lst:handler_fun}.

    \bigskip
    \begin{code}[caption=Simple serverless.yml file,
        label={lst:simple_sls_yml}, language=yaml]
org: my-company-org
app: chat-app
service: users
provider:
  name: aws
  runtime: nodejs12.x
functions:
  usersCreate:
    handler: handler.usersCreate
    events:
      - http: post users/create
    \end{code}

    \begin{code}[caption=Simple handler function,label={lst:handler_fun}]
async function usersCreate(event, context) {
  const user = {
    name: 'sample_name',
    surname: 'sample_surname'
  }
  await mockDb.createUser(user)
  return {
    statusCode: 200,
    body: JSON.stringify({user})
  }
}
    \end{code}

    \begin{figure}
        \begin{minipage}{\linewidth}
            \dirtree{%
                .1 ./.
                .2 handler.js.
                .2 serverless.yml.
            }
        \end{minipage}
        \caption{Simple Serverless project structure}
        \label{fig:sls_project_structure}
    \end{figure}

    Serverless is flexible and does not force a fixed structure of the project, that
    task is up to the developer.
    Defined that structure, the service can be deployed using the Serverless CLI, on
    the chosen provider, as shown on listing \ref{lst:sls_deploy}.

    \bigskip
    \begin{code}[caption=Deploy command, label={lst:sls_deploy}, language=shell]
$ serverless deploy
Serverless: Stack update finished...
Service Information
service: users
stage: dev
region: us-east-1
stack: users-dev
resources: 12
api keys:
  None
endpoints:
  POST - https://.../dev/users/create
functions:
  usersCreate: users-dev-usersCreate
layers:
  None
    \end{code}

    The deploy command creates the necessary aws resources, in this case they are:
    a lambda function corresponding to the usersCreate function and an api gateway to
    handle http requests.
    It is then possible to test the newly created resource by making requests to the
    url returned by the CLI, specifying the resource path /users/create.
    It is possible to invoke online functions also directly from the CLI,
    specifying the identifier of the function used in the serverless.yml file, as shown
    on listing \ref{lst:sls_invoke}

    \bigskip
    \begin{code}[caption=Invoke command, label={lst:sls_invoke}, language=shell]
$ serverless invoke -f usersCreate
{
    "statusCode": 200,
    "body": "{\"user\":{\"name\":\"sample_name\", ...}}"
}
    \end{code}

    The development and deploy process shown for a service with a single function
    remains the same as the service complexity grows, in particular it is possible to
    modify and deploy a single function at a time, since each function has its own
    resource associated.
    This process gets along with the previously described micro services architecture.

    \subsection{Advantages}
    The main advantages of using the Serverless framework are:
    \begin{itemize}
        \item Provider agnostic: the framework aims to be independent from the chosen
            cloud provider, thus avoiding vendor lock-in. In practice this feature is
            not achieved completely, as the configuration file serverless.yml may be
            different across providers. However the main structure remains the same,
            and that simplify providers migration.
        \item Simplified development: the CLI commands simplify the development process,
            from the deploy from the testing of the deployed functions.
        \item Extensible: is possible to develop plugins that integrate with the
            CLI commands lifecycle, increasing their functionalities.
        \item Dashboard: the hosted dashboard allow monitoring and tracing of the
            deployed functions and services.
    \end{itemize}

    \subsection{Disadvantages}
    \label{subsec:sls_disadvantage}
    The main advantages of using the Serverless framework and the Serverless paradigm are:
    \begin{itemize}
        \item Compilation of the configuration file may become tedious as the project grows.
        \item The framework is extremely flexible regarding the project structure and
            that is an advantage, however this can be also a drawback as it's up to the
            developer to find a suitable structure, and this means less time spent on
            business related tasks.
        \item Unit testing: it is possible to test a deployed function easily, however
            for big projects, where it's necessary to test a lot of functions, this may
            become cumbersome.
        \item Resource threshold: for projects created with Aws, a single
            serverless.yml file may create up to 200 resources, and if exceeded
            the deploy operation fails. Since each function is responsible for the
            creation of about 10 resources, is very easy to exceed this limit.
            The only solution so solve this problem is to split the functions across
            multiple services, hence different serverless.yml configuration files.
        \item Cold start: inherent overhead of the current implementation of the
            serverless paradigm. Since each function is executed only in response to
            an event, a certain amount of time is required for resources initialization.
    \end{itemize}

    \section{Conclusions}
    Each cloud model presented has its own strength and drawbacks, depending on the
    needs of the wanted goal. Favouring as selection criteria, solutions that present
    major advantages in terms of scalability, cost efficiency and speed of development,
    has been decided to favour the Serverless option.
    The main cloud providers offering this kind of service, as previously stated,
    are: Aws, with its Lambda service, Microsoft, with Azure Functions, and Google,
    with Cloud Functions. Each provider offer different configurations, with different
    pricing, based on memory, CPU, and execution time as parameters, as shown on
    \ref{table:cloud_providers_offer}.
    In the literature there are several documents comparing the various services
    side by side exhaustively \cite{sls_providers_comparison}.
    For the project subject of this document has been chosen Aws as the main provider,
    as the most mature platform meeting the project's needs. In particular it
    providers the following advantages with respect to the competitors \cite{sls_providers_comparison}:
    \begin{itemize}
        \item Cold start (\ref{table:cloud_providers_cold_start})
        \item Overall maturity
        \item Performance consistency
        \item Scalability
    \end{itemize}

    \begin{table}
        \centering
        \begin{tabularx}{0.8\textwidth}{
                | >{\raggedright\arraybackslash}X
                | >{\centering\arraybackslash}X
                | >{\centering\arraybackslash}X
                | >{\centering\arraybackslash}X |
            }

            \hline
            & \textbf{AWS} & \textbf{Azure} & \textbf{Google} \\
            \hline\hline
            Memory (MB) & 64 * k (k = 2, 3, ..., 24) & 1536 & 128 * k (k = 1, 2, 4, 8, 16) \\
            \hline
            CPU & Proportional to Memory & Unknown & Proportional to Memory \\
            \hline
            Language & Python Nodejs Java, and others & Nodejs Python, and others & Nodejs \\
            \hline
            Runtime OS & Amazon Linux & Windows 10 & Debian 8 \\
            \hline
            Local disk (MB) & 512 & 500 & > 512 \\
            \hline
            Run native code & Yes & Yes & Yes \\
            \hline
            Timeout (second) & 300 & 600 & 540 \\
            \hline
            Billing factor & Execution time, Allocated memory & Execution time,
                Consumed memory & Execution time, Allocated memory, Allocated CPU\\
            \hline
        \end{tabularx}
        \caption{Cloud providers configuration \cite{sls_providers_comparison}}
        \label{table:cloud_providers_offer}
    \end{table}

    \begin{table}
        \centering
        \begin{tabularx}{0.8\textwidth}{
                | >{\raggedright\arraybackslash}X
                | >{\centering\arraybackslash}X
                | >{\centering\arraybackslash}X
                | >{\centering\arraybackslash}X
                | >{\centering\arraybackslash}X |
            }

            \hline
            Provider-Memory & Median & Min & Max & STD \\
            \hline\hline
            AWS-128 & 265.21 & 189.87 & 7048.42 & 354.43 \\
            \hline
            AWS-1536 & 250.07 & 187.97 & 5368.31 & 273.63 \\
            \hline
            Google-128 & 493.04 & 268.5 & 2803.8 & 345.8 \\
            \hline
            Google-2048 & 110.77 & 52.66 & 1407.76 & 124.3 \\
            \hline
            Azure & 3640.02 & 431.58 & 45772.06 & 5110.12 \\
            \hline
        \end{tabularx}
        \caption{Cloud providers Cold start (in ms) \cite{sls_providers_comparison}}
        \label{table:cloud_providers_cold_start}
    \end{table}

\end{chapter}
        % \begin{chapter}{Tools}

    An important process in the software development is the choice of the right
    tools, in order to achieve simplicity and efficiency for both development
    process and the project itself.
    In this chapter will be described the main tools used during the development
    of Restlessness and its deployment to make it available for everyone.

    \section{JavaScript}
    JavaScript is a lightweight interpreted programming language. Interpreted means
    that the code is read top to bottom, and the result of the running code is
    immediately returned. Interpreted programming languages are opposed to compiled
    one, where the code is transformed into a binary format that can be directly
    executed \cite{what_is_js}.
    Although JavaScript was born as a language limited to client side programming,
    exploiting an engine directly incorporated into the Web browser, with the
    introduction of Node.js has become possible to use this language also for backend
    programming, and in general in contexts outside of the browser.
    Node.js is a JavaScript runtime based on the V8 engine, core engine of the popular
    Chrome browser \cite{node_org}.
    A key characteristic and one of the main strength of JavaScript with respect
    to other programming languages is its asynchronous nature, that allows having
    non-blocking I/O. As a consequence of this, the code runs on a single thread,
    based on a LIFO queue (Last In, First Out) continuously checked by the so called
    Event Loop.
    As shown on \ref{fig:event_loop}, operations regarding File System, Network or
    Database access are executed separately, and only once completed are inserted
    again into the queue, to handle their result. Meanwhile other queued code
    is executed by the only present thread.

    \begin{figure}
        \centering
        \includegraphics[width=\linewidth]{source/images/nodejs_event_loop.png}
        \caption{The Event Loop \cite{node_event_loop_2}}
        \label{fig:event_loop}
    \end{figure}

    Being single threaded is a useful limitation as it's not possible to incur
    into concurrency issues.
    This peculiarities make JavaScript well suited for the so called real-time
    applications (RTAs), that is applications that have to process a high volume
    of short messages requiring low latency, and so they require a highly scalable
    solution. Conversely due to its single threaded nature, JavaScript is not
    recommended for CPU-heavy jobs, as the Event Loop would be stuck on a single
    operation \cite{node_event_loop}\cite{node_on_backend}.

    Another advantage of JavaScript, especially after the release of Node.js, is
    the possibility to use the language for both frontend and backend in the context
    of Web development, creating a seamless experience for developers.

    JavaScript is a dynamically typed language, which means that it's not necessary
    to explicitly mention the type of data a variable holds, as that type can change
    dynamically as the content of the variable changes (\ref{lst:dynam_type}).

    \begin{code}[caption=Dynamically typed variables, label={lst:dynam_type}]
let a = "Hello World!"
a = 42
    \end{code}

    This feature of the language gives a lot of flexibility to developers, however
    as the project complexity grows it can quickly become a downside.
    For this reason, in 2012 Microsoft released an open source language called
    Typescript, a superset of JavaScript that enable static type checking.
    Being a superset, any JavaScript code is also valid Typescript code, enabling
    a gradual integration for already existing code bases.
    The Typescript compiler is specifically a transpiler, or a compiler that takes
    source code as input and produces other source code as output, in this case
    JavaScript code. The compiler will point the errors it encounters, but it does
    not prevent the code to be run, hence it behaves like a spellchecker for the code.
    Typescript can also infer variables type from their usage, reducing the effort
    needed to enable static type checking from the developer
    \cite{typescript_lang}\cite{typescript}.
    Keeping in mind the described strengths of the JavaScript environment, it has been
    decided to use it as the main language for the development of the Restlessness
    framework.

    \bigskip
    \begin{code}[caption=Static type checking on Typescript, label={ts_static}]
interface Student {
  name: string
  graduationYear: number
}

const aStudent: Student = {
  name: 'Arthur Dent',
  graduationYear: 2020
}

aStudent.graduationYear = '2020'    // Invalid
aStudent.graduationYear = 2021      // Valid
    \end{code}

    \section{Npm}
    The strengths of the JavaScript ecosystem are further increased by the presence
    of Npm, shorthand for Node Package Manager, which is the official package manager
    for Node.js. Npm rely on the CommonJs modules specification \cite{common_js},
    which defines a convention for the JavaScript module ecosystem.
    The main components of Npm are:
    \begin{itemize}
        \item Npm registry: modules can be published to it or installed from it.
            The official and main registry is available at the address
            \url{https://npmjs.org}
        \item \textit{npm} CLI: the command line tool from which is possible to interact
            with the registry, with operations like publishing or installing packages.
        \item \textit{package.json}: a configuration file, in the Json format
            \cite{jsoniso}, that must be present for both modules that are published
            into the registry and modules that use other modules from the registry
            as dependencies. It contains projects informations, such as name and
            version, and a list of other modules, on which the project depends on.
        \item \textit{node\_modules}: an automatically created folder that contains
            all the projects dependencies. At runtime Node.js looks for modules
            in this folder.
    \end{itemize}

    Listing \ref{lst:add_module} shows the \textit{package.json} of a simple module,
    while \ref{lst:add_fn} shows the definition of a function, on that module,
    exported using the CommonJs specification. To publish the package on the Npm
    registry is possible to invoke the \textit{publish} command on the \textit{npm}
    CLI. With the \textit{install} command that same package is installed as
    dependency under the \textit{node\_modules} folder, and can be used as shown
    on listing \ref{lst:add_require}

    \bigskip
    \begin{code}[caption=A simple \textit{package.json},
        label={lst:add_module}, language=json]
{
    "name": "add_module",
    "version": "1.0.0",
    "description": "Simple module example",
    "main": "index.js",
    "author": "Arthur Dent",
    "license": "ISC"
}
    \end{code}

    \bigskip
    \begin{code}[caption=CommonJs module definition, label={lst:add_fn}]
// index.js
function add(n1, n2) {
  return n1 + n2
}

module.exports = add
    \end{code}

    \bigskip
    \begin{code}[caption=CommonJs module usage, label={lst:add_require}]
const add = require('./add.js')
    \end{code}

    The Npm ecosystem has been used extensively during the development of Restlessness,
    for its dependencies and for making it available on the registry.
    Furthermore, the developed framework uses a feature of Npm called Scoped Packages
    \cite{npm_scoped_packages}, which allows to group related packages together
    under a common scope, acting as a namespace. Restlessness packages are available
    under the \textit{@restlessness/} scope.

    \section{Github}
    \subsection{Git}
    Git is an Open Source Distributed Version Control System, in particular:
    \begin{itemize}
        \item Control System: Git is a content tracker, it can be used to store
            content, which generally is code.
        \item Version: the tracked content is subject to continuous change, often
            this changes are added in parallel. Git helps handling this by maintaining
            a history of all changes.
        \item Distributed: Git is based on remote and local repositories, the first
            one stored in a server, while the latter is stored in the developer
            computer, and both contains the full history information.
    \end{itemize}

    Git is useful to track code changes in all cases, but it's absolutely necessary
    to avoid conflicts when multiple developers work in parallel on a single codebase.
    The main concepts introduced by Git are:
    \begin{itemize}
        \item Commit: the main unit representing content modification.
        \item Branches: allow working simultaneously at the codebase, making
            different modifications.
        \item Push/Pull: operations that allow synchronization between the remote
            repository and the local one.
        \item Merge: operation that integrate the modification made on a branch
            into another branch.
        \item Tag: a string identifier assigned to a specific commit, useful to
            reference a particular version of the project (e.g. a simple tag is
            \textit{v1.0.2}).
    \end{itemize}

    \begin{figure}
        \centering
        \includegraphics[width=\linewidth]{source/images/rln-git-history.png}
        \caption{Section of Restlessness history}
    \end{figure}

    With these concepts it is possible to work on each feature independently from
    others, integrating it only when it reaches an appropriate stability level.
    The strategy adopted with the developed framework has been to create branches
    with the \textit{feature/} prefix for new functionalities or improvement of
    existing ones, and the \textit{fix/} prefix for correction of bugs, followed
    by the name of the specific feature of fix.

    \subsection{Github features}
    Github is a web based platform providing all functionalities offered by the Git
    system plus additional DevOps features, with the main ones used during the
    development of Restlessness being: Issues, Pull Requests and Projects.

    \subsubsection{Issues}
    Issues are Github feature that helps to keep track of tasks, bugs, enhancements
    or any kind of modification to the project. They are characterized by a title,
    that gives an immediate feedback about what is the reason of the Issue, and an
    optional description, with more specific and technical information, as shown
    on figure \ref{fig:rln_issue}. Each Issue can be assigned to one or more
    collaborators, responsible for having it solved. This tracking system is focused
    on collaboration, as it is possible to comment and discuss about the Issue with
    other collaborators, also referencing other resources, which can be other Issues
    or code sections.
    As the project grows so does the number of Issues, and so it becomes important to
    keep them organized. This is made possible by using Labels and Milestones.
    Both allow to group Issues according to a common characteristic, but with a
    different granularity \cite{github_issues}.
    The first one allows a more specific grouping, with the main ones defined for
    Restlessness being:
    \begin{itemize}
        \item \textit{enhancement}: A new feature, or a request for a new feature.
        \item \textit{bug}: A problem in the project functionalities.
        \item \textit{documentation}: Improvements or additions to documentation.
        \item \textit{tests}: Testing related Issues
        \item \textit{good first issue}: Being the framework Open Source, also
            external people can contribute to it, this Label marks simple and easy
            Issues that can be managed also by newcomers.
        \item Packages specific Issues: Restlessness adopt a monorepo  strategy
            \cite{monorepo}, having all provided packages under the same repository,
            so it has been defined a Label for each package, such as: \textit{CORE},
            \textit{CLI}, \textit{AUTH-cognito} and \textit{DAO-mongo}.
    \end{itemize}
    The latter instead group together Issues linked together from a temporal point
    of view, typically a version release or a planned Sprint if following the agile
    methodology \cite{agile}. With the Restlessness framework it has been opted for
    the first option.


    \begin{figure}
        \centering
        \includegraphics[width=\linewidth]{source/images/github-rln-issue.png}
        \caption{An Issue on the Restlessness project}
        \label{fig:rln_issue}
    \end{figure}

    \begin{figure}
        \centering
        \includegraphics[width=\linewidth]{source/images/github-rln-issues-list.png}
        \caption{List of closed Restlessness Issues}
        \label{fig:rln_issues_list}
    \end{figure}

    \subsubsection{Pull Requests}
    An important process when multiple developers collaborate on a single project
    are code reviews, as having project's modification verified by more than one
    person reduces the risk of finding bugs, typos and critical problems later.
    Pull Requests are a feature of Github that enable this process. With it a
    collaborator proposes its changes while another one accepts or rejects the request.
    It is possible to discuss on the specific request, referencing other resources,
    commenting on code or requesting modification on the proposed changes, as it
    happens for Issues.
    When a Pull Request is created the author chooses a target branch on which to
    integrate its proposed changes, and once the request is accepted those changes
    are merged into the target branch, and the Pull Request is considered closed,
    as shown on figure \ref{fig:rln_pull_request}.

    \begin{figure}
        \centering
        \includegraphics[width=\linewidth]{source/images/rln-pull-request.png}
        \caption{An approved Pull Request on the Restlessness project}
        \label{fig:rln_pull_request}
    \end{figure}

    \subsubsection{Projects}
    Projects is a recently added Github feature with the purpose of further improve
    organizing and distributing tasks and work. From the Projects page it is possible
    to define custom columns in which assign different tasks, which can be Issues,
    Pull Requests or simple Notes. As shown in figure \ref{fig:rln_project_board},
    for Restlessness has been defined tree columns: \textit{To do},
    \textit{In Progress} and \textit{Done}. This way it is immediately visible which
    tasks need to be done, are under development or are already completed.

    \begin{figure}
        \centering
        \includegraphics[width=\linewidth]{source/images/rln-github-project-board.png}
        \caption{Github Projects board on Restlessness}
        \label{fig:rln_project_board}
    \end{figure}

    \bigskip
    \bigskip
    \noindent
    Being the developed framework Open Source, it is available for consultation,
    modification and improvement on Github, as well as this document, on the
    following addresses:
    \begin{itemize}
        \item Restlessness: \url{https://www.github.com/getapper/restlessness}
        \item Thesis: \url{https://www.github.com/androsanta/Thesis}
    \end{itemize}

    \section{CircleCi}

    \subsection{CI/CD}
    Continuous Integration is a practice that encourages developers to integrate
    their code changes early and often, into the main and stable version of the
    project, which for a git based project is the \textit{master} branch.
    Each code integration triggers an automated build and test, that if failed can
    be repaired quickly.
    The main advantage of using this approach is the early bug detection, which
    as consequence will result in an overall reduced bug count and reduced
    maintenance. Moreover once set, the CI process does not add any overhead to
    the development as it is completely automated.
    The CI approach is oftentimes related to another approach, which is the
    Continuous Delivery, defined as:

    \enquote*{%
        Continuous Delivery (CD) is a software engineering approach in which teams
        keep producing valuable software in short cycles and ensure that the
        software can be reliably released at any time.%
    } \cite{continuous_delivery}

    \subsection{The platform}
    CircleCi is an online platform that provides services for implementing
    Continuous Integration and Continuous Delivery (CI/CD) on software projects.
    It can be configured to access the source code repository on Github, and after
    that each commit can trigger an automated build, test and deploy task.
    Those automated tasks are performed inside a clean container or Virtual Machine,
    ensuring a reproducible environment.

    \begin{figure}
        \centering
        \includegraphics[width=\linewidth]{source/images/circle-ci.png}
        \caption{CircleCi flow \cite{circle_ci_official}}
        \label{fig:circle_ci_structure}
    \end{figure}

    \noindent
    The main concepts introduced by the platform are:
    \begin{itemize}
        \item Configuration: All processes are orchestrated through a single file
            called \mbox{\textit{config.yml}}, in the \href{https://yaml.org/}{Yaml}
            format, and placed under a folder called \textit{.circleci} at the root
            of the project.
        \item Orbs: Reusable snippets of code that help automate repeated processes
        \item Jobs: Building blocks of the configuration file, they are a collection
            of steps, which run commands or scripts as specified. Each Job is run
            in a unique executor.
        \item Executor: The container or Virtual Machine that run each Job.
            It is possible to chose between \href{https://www.docker.com/}{Docker}
            containers, Virtual Machines running Linux, Windows or MacOS.
        \item Steps: Actions that need to be taken to complete a Job. It can be
            any kind of executable command.
        \item Workflows: They define a list of Jobs with their run order, and
            concurrency.
    \end{itemize}

    For the Restlessness development has been chosen the popular containerization
    solution called \href{https://www.docker.com/}{Docker}, in particular a Node.js
    based container, as shown on listing \ref{lst:ci_executor}:

    \bigskip
    \begin{code}[caption=Reusable executor definition, label={lst:ci_executor},
        language=yaml]
executors:
  node12:
    docker:
      - image: circleci/node:12.9.1
    \end{code}

    As previously said the framework adopt a monorepo structure, so it has been
    necessary to define multiple Workflows, one for each package. Each Workflow
    defines two parallel Jobs, for testing and publishing on the Npm registry.
    Figure \ref{fig:ci_workflow_diagram} show the described structure for two Restlessness
    packages, and it is possible to notice that each Job run in its own container,
    in parallel and independently from the others

    \begin{figure}
        \centering
        \includegraphics[width=\linewidth]{source/diagrams/ci_workflow.png}
        \caption{CircleCi workflows for Restlessness}
        \label{fig:ci_workflow_diagram}
    \end{figure}

    To perform the Steps shown on \ref{fig:ci_workflow_diagram} has been defined
    reusable commands, with the main one being:
    \begin{itemize}
        \item \textit{install\_packages}: Install dependencies as specified by the
            \textit{package.json}.
        \item \textit{deps\_and\_tests}: Install dependencies and run tests as
            specified by the \mbox{\textit{package.json}}.
        \item \textit{npm\_publish}: Publish the package on the Npm registry.
    \end{itemize}

    According to the Continuous Delivery approach the publish operation is triggered
    manually by performing a git tag on a specific repository commit, following the
    format: package name, followed by \textit{/v} and the semantic version of the
    package (e.g. \textit{@restlessness/core/v1.0.2}). A custom script takes care
    of extracting the version information and setting it on the correct
    \textit{package.json}, where is read from the npm publish command.

    Although CircleCi offers its own website from which is possible to check Workflows
    execution, errors and details of every operation, it offers also a Github plugin,
    that is able to show Workflows result directly on commits or Pull Requests, as
    shown on \ref{fig:ci_github_integration}. The integration between the two
    services has simplified the development workflow of Restlessness, and it adds to
    the already described advantages of adopting a CI/CD approach.



    \begin{figure}
        \centering
        \includegraphics[width=\linewidth]{source/images/ci-github-integration.png}
        \caption{CircleCi Workflows seen from Github}
        \label{fig:ci_github_integration}
    \end{figure}

    \section{AWS}
    Amazon Web Services is a cloud platform offered by \textit{Amazon.com, Inc}
    which, among its various services, also provides serverless computing options.
    Although one of the purpose of using the Serverless Framework is to abstract
    the underlying infrastructure details of the platform, those details are needed
    to develop a framework such as Restlessness, that has to interact with the
    platform at a lower level to provide its functionalities.

    Here is a list of the main services used by Serverless and Restlessness on
    behalf of the user and also used during Restlessness development:
    \begin{itemize}
        \item Lambda: The compute service providing the serverless functionalities.
            A Lambda function contains the code written by the developer.
        \item API Gateway: A service that creates a connection point between
            external requests and other internal services, such as a Lambda functions.
        \item S3: Acronym for Simple Storage Service, provides object storage.
            Resources are organized in containers called Buckets.
        \item CloudFormation: A service that allow to model infrastructure as code.
            Each CloudFormation configuration corresponds to a resource called
            CloudFormation Stack, containing the description of other resources,
            such as AWS Lambda functions, API Gateway, and how such resources may
            interact.
        \item CloudWatch: A services for monitoring and observability.
        \item IAM:Acronym for Identity and Access Management, enables the management
            of AWS resources access.
    \end{itemize}

    \subsubsection{Resource creation during deploy}
    During the deployment of a Serverless service the user code and its dependencies
    are packaged into a zip artifact. It then begins the remote resource creation of
    a CloudFormation Stack and an S3 Bucket. Once that resources initialization has
    been completed, the CloudFormation configuration and the zip artifact are uploaded
    and saved into the S3 Bucket and that operation is followed by the creation of
    all resources defined on the CloudFormation Stack. Those operations are shown
    on figure \ref{fig:sls_deploy_on_aws}

    \begin{figure}
        \centering
        \includegraphics[width=8cm]{source/diagrams/sls_deploy_on_aws.png}
        \caption{Resources creation on Serverless deploy for a \textit{User} service}
        \label{fig:sls_deploy_on_aws}
    \end{figure}

    \subsubsection{Lambda function invocation through an API Gateway}
    \label{subsec:lambda_invocation}

    Figure \ref{fig:simple_lambda} shows the simplest possible case of execution
    flow of an http request, handled by an API Gateway, and forwarded to the Lambda
    function mapped to the user specified endpoint path.

    \begin{figure}
        \centering
        \includegraphics[width=\linewidth]{source/diagrams/lambda_invocation.png}
        \caption{Simple Lambda function execution through an API Gateway}
        \label{fig:simple_lambda}
    \end{figure}

    A more complex case is given when implementing user Authentication and hence
    restricting Lambda execution. The Authentication process is made simple by
    delegating the operation of granting or denying Authentication to a Lambda
    function, called Lambda Authorizer \cite{aws_api_gateway_doc}, as shown on
    figure \ref{fig:lambda_with_auth}. There can be two types of Lambda Authorizers:
    \begin{itemize}
        \item TOKEN: the Lambda receives the caller's identity in a bearer token.
        \item REQUEST: the Lambda receives the caller's identity in a combination
            of headers and query string parameters.
    \end{itemize}

    \begin{figure}
        \centering
        \includegraphics[width=\linewidth]{source/diagrams/lambda_authorizer.png}
        \caption{Lambda Authorizer function, based on TOKEN identity}
        \label{fig:lambda_with_auth}
    \end{figure}

    The API Gateway forward the request to the specified Lambda Authorizer, that
    checks the caller's identity and generates an Authentication Policy, which is
    an object that states which resources the user is authorized to access.
    The Policy is then cached to improve performance on subsequent requests, and
    if it the access request is granted, the flow proceed as in the previously
    described case.

    Serverless abstract this structure by allowing to specify a function as
    Authorizer of another function, as shown on listing \ref{lst:sls_auth}, where
    the \textit{getUsers} function is executed only if the function \textit{auth}
    grants access.

    \bigskip
    \begin{code}[caption=Authorizer definition on Serverless, label={lst:sls_auth},
        language=yaml]
functions:
  auth:
    handler: auth.customAuth  # auth.js
  getUsers:
    handler: users.getUsers   # users.js
    events:
      - http:
        path: hello
        method: get
        authorizer: auth
    \end{code}

    \section{React}
    An important part of the Restlessness framework is its Graphical User Interface,
    which is the main interaction point for the user. The Frontend development,
    specifically toward Web Interfaces, can count on the presence of several
    libraries and frameworks based on the JavaScript language. For the development
    of Restlessness it has been chosed the popular library React, due to its simplicity,
    and effectiveness.

    React is an Open Source JavaScript library that implements the concept of
    virtual DOM (Document Object Model) \cite{dom_standard}. The browser creates
    a DOM object at page loading, and then each Html object inside the DOM can be
    manipulated using JavaScript functions, giving the user an immediate feedback.
    React instead adopt a different approach by creating a virtual DOM alongside
    the real one. The virtual DOM is not directly synched with the real one, so
    it can be modified much faster, not having to reflect those modification on
    the screen. After those virtual DOM updates are created using the React api,
    the new istance of the virtual DOM is compared to the previous one, allowing
    to reflect the update on the real DOM only for the elements that actually
    change. The library allows to create a structure based on reusable component,
    obtaining a scalable structure, and is particularly suited for SPA (Single
    Page Applications) \cite{react_js}.
    The library also introduced a new syntax, named JSX (JavaScript XML), and
    listing \ref{lst:react_example} show the definition of a React component.

    \bigskip
    \begin{code}[caption= React component definition,label={lst:react_example}]
import React from 'react';

const Card = ({name}) => {
  return (
      <div>{name}</div>
  );
};
    \end{code}

\end{chapter}
        % \begin{chapter}{Restlessness}
    \label{chap:restlessness}

    \begin{figure}
        \centering
        \includegraphics[width=5cm]{source/images/restlessness_logo.png}
        \caption{The Restlessness logo}
    \end{figure}

    The framework is composed by different components, listed here:
    \begin{itemize}
        \item Command Line Interface: together with the Web Interface, this is the
            main component with which users interact the most. It is available as
            the @restlessness/cli package on npm.
        \item Restlessness frontend: Web Interface with which it is possible to
            create resources and manage the project. It is part of the CLI.
        \item Restlessness backend: api service running locally, created with the
            Restlessness framework itself. It is used by the Web Interface to
            provides its functionalities.
        \item Restlessness core: core package of the framework, it contains all the
            classes and functions that provides the framework functionalities. It is
            available as @restlessness/core package on npm.
    \end{itemize}

    \section{Project creation}
    \label{sec:rln_project_creation}

    The Restlessness CLI is available for installation on the npm platform, as described
    on chapter \ref{chap:development}. % @TODO ref - be more specific after drafting the chapter
    Once installed, the first step toward using the framework is
    the creation of a new project, and that is possible using the 'new' command, as shown
    on listing \ref{lst:rln_command_new}.
    % description: creation of a new project, named rln_project
    \begin{lstlisting}[caption=New command, label={lst:rln_command_new}]
$ restlessness new rln_project
    \end{lstlisting}
    Once the command has finished, a new folder has been created, with a completely
    structured restlessness project, as can be see in figure \ref{fig:sample_rln_project_folder}.

    \begin{figure}
        \caption{Sample Restlessness project structure}
        \label{fig:sample_rln_project_folder}
        \begin{minipage}{\linewidth}
            \dirtree{%
                .1 ./.
                .2 .restlessness.json.
                .2 configs/.
                .3 authorizers.json.
                .3 daos.json.
                .3 default-headers.json.
                .3 endpoints.json.
                .3 envs.json.
                .3 models.json.
                .3 schedules.json.
                .2 envs/.
                .3 .env.locale.
                .3 .env.production.
                .3 .env.staging.
                .3 .env.test.
                .2 serverless-services/.
                .3 offline.json.
                .3 shared.json.
                .2 src/.
                .3 exporter.ts.
                .3 schedulesExporter.ts.
            }
        \end{minipage}
    \end{figure}

    The sample project shown in figure \ref{fig:sample_rln_project_folder} however,
    does not include all generated files, as some of them are not strictly part of the
    framework, but are required from other used tools, in particular:
    \begin{itemize}
        \item .eslintrc.json: configuration file of the linter
            \href{https://eslint.org/}{eslint}.
        \item .gitignore: list intentionally ignored files from the git tracking system.
        \item package.json: entry point of every npm project, it lists the project
            dependencies, as well as other project related information, such as
            the project name and version.
        \item package-lock.json: npm generated file, contains a snapshot of the
            version of all dependencies, with the goal of obtaining reproducible builds.
        \item tsconfig.json: configuration file for the Typescript compiler.
    \end{itemize}

    The first noticeable difference with respect to a plain serverless project is the
    lack of a serverless.yml (or serverless.json) file under the root.
    % @todo threshold limitation ref from chap:into and maybe also chap:aws
    In fact, due to the resource threshold limitation imposed by Aws, has been decided
    to let the framework manage the presence of multiple services inside a single
    project, so setting up a structure that is standard and micro services oriented.
    Following this structure, all serverless.yml file correspondents to the various
    services, are located under the serverless-services folder.
    Has been decided to format those configuration files using Json, instead of Yaml,
    to simplify their handling and modification by the framework, given that Typescript
    handle Json files and objects natively.
    After creation, the sample Restlessness project already contains two services,
    named shared and offline, and they are required for the framework to work.

    The shared service will contains all shared resources that can be used by all
    the other services. This is the case for the api gateway (@todo ref aws api
    gateway description) because, since it is responsible for handling http requests,
    it is convenient to have a single url for all services, rather than one for
    each service. Other shared resources may be simple functions or authorizers.
    The offline service is required to handle local development, as it contains the
    resource definition of all services.
    % @todo parlare in maniera più approfondita del fatto che il servizio offline
    % deve essere trasparente per l'utente e under the hood il framework si occupa
    % di mantenere offline in sync con gli altri servizi

    Other created files are: configuration files, under the config folder, environment
    files, containing environment variables for different deployments, source code,
    under the src folder, and a .restlessness.json file, used to store project related
    information needed by the framework.

    \section{Local development}
    \label{sec:local_dev}
    The local development requires the presence of different processes, and as
    previously said, framework's side are necessary a web interface and an api
    service, and also a the project's process to be able to test it.
    The CLI handles those 3 processes through a single process, named dev (listing
    \ref{lst:rln_command_dev}).
    In particular, both the project's process and Restlessness backend, are executed
    using the Serverless plugin serverless-offline, which allow simulating an api
    gateway, effectively creating a local http server.
    Instead for the frontend process has been used the npm package serve, through
    which is possible to create an http server that serve static files.
    Furthermore the dev command takes care of executing those processes following the
    dependency order, which is: Restlessness backend, frontend and finally the
    project's process.

    % @todo schema per il processo rln che fa partire gli altri 3? see plantuml

    Another task of the dev command is to implement inter process communication between
    itself and the backend process. This is necessary as when resources are created,
    for example endpoints or schedules, the corresponding files need to be compiled by
    typescript and also the serverless-offline plugin needs to be restarted for those
    resources to be available from the http server.
    % @todo forse anche qua ci sta uno schemino

    As shown on listing \ref{lst:rln_command_dev}, the command receives the environment
    name in input, as it takes care of loading the corresponding environment variables
    from the folder .envs, as explained on section \ref{sec:env_vars}.

    % @todo show also command output
    \begin{lstlisting}[caption=Dev command, label={lst:rln_command_dev}]
$ restlessness dev locale
    \end{lstlisting}

    \section{Resource creation}
    The Web Interface looks like in the figure \ref{fig:rln_web_interface}, and provides
    some project details, such as serverless organization, application (section
    \ref{sec:serverless_framework}), and finally the aws data center region to which
    the project will be deployed.
    The main functionalities are then available through some shortcuts, that allow
    creating and consulting resources, such as endpoints, schedules, services and models.

    \begin{figure}
        \centering
        % @todo \includegraphics[width=\linewidth]{source/images/}
        \caption{Restlessness Web Interface}
        \label{fig:rln_web_interface}
    \end{figure}

    Being Restlessness a framework for serverless services, the primary resource that
    can be defined are functions, and at the moment it is possible to define two type
    of functions, based on the event that triggers them. They are endpoints, for http
    event, and schedules, for programmed events, such as cron jobs.

    \subsection{Endpoints}
    \label{subsec:endpoints}
    It is possible to create an endpoint from the Web Interface, by specifying the
    following fields, as shown on figure \ref{fig:wi_create_endpoint}:
    \begin{itemize}
        \item Service: the service to which the function must be associated.
        \item Route: the path corresponding to the serverless function.
        \item Method: the http method.
        % \item Warmup enabled: @todo
        % \item Daos: @todo
        \item Authorizer: this optional field sets a further function, that
            perform the authorization operation, granting or denying access to
            the specified function. %, as explained on (@todo ref).
    \end{itemize}

    \begin{figure}
        \centering
        % @todo \includegraphics[width=\linewidth]{source/images/}
        \caption{Creation of and endpoint}
        \label{fig:wi_create_endpoint}
    \end{figure}

    During the endpoint creation, the framework takes care of saving the provided
    information on the configuration file config/endpoints.json, and to create code
    template for the development of the corresponding function.
    As shown on figure \ref{fig:new_endpoint_folder_structure}, has been created a
    folder under src/endpoints, using the notation http method plus normalized value
    of the http path.

    \begin{figure}
        \caption{Structure of a new endpoint folder}
        \label{fig:new_endpoint_folder_structure}
        \begin{minipage}{\linewidth}
            \dirtree{%
                .1 ./.
                .2 src/.
                .3 endpoints/.
                .4 post-users.
                .5 handler.ts.
                .5 index.ts.
                .5 index.test.ts.
                .5 interfaces.ts.
                .5 validations.ts.
                .3 exporter.ts.
                .3 schedulesExporter.ts.
            }
        \end{minipage}
    \end{figure}

    The developer can then code the function in handler.ts, which already contains a
    template (listing \ref{lst:handler_ts}) and define the validation object in
    validations.ts (listing \ref{lst:validations_ts}).
    It is also possible to exploit the Typescript functionalities, defining the various
    interface for the request, response and query parameters objects, all under the
    interfaces.ts file (listing \ref{lst:interfaces_ts}).
    The actual function entry point that will be executed once deployed is defined
    in the file index.ts (listing \ref{lst:index_ts}). This function is created binding
    the function LambdaHandler input with the handler function and validation object.
    LambdaHandler is a core function of the framework, its purpose is to parse the
    request payload and or query parameters, load the environment variables (section
    \ref{sec:env_vars}) and execute the lifecycle hooks of the installed addons (chapter
    \ref{chap:extensions}).
    After those operation the LambdaHandler execute the actual handler function.

    \begin{lstlisting}[caption=handler.ts content, label={lst:handler_ts}]
export default async (req: Request) => {
    try {
        const {
            validationResult,
            payload,
        } = req;

        if (!validationResult.isValid) {
            return ResponseHandler.json({
                message: validationResult.message
            }, StatusCodes.BadRequest);
        }

        return ResponseHandler.json({});
    } catch (e) {
        console.error(e);
        return ResponseHandler.json(
            {}, StatusCodes.InternalServerError);
    }
};
    \end{lstlisting}

    \begin{lstlisting}[caption=index.ts content, label={lst:index_ts}]
export default LambdaHandler
    .bind(this, handler, validations, 'postUsers');
    \end{lstlisting}

    \begin{lstlisting}[caption=validations.ts content, label={lst:validations_ts}]
const queryStringParametersValidations =
(): YupShapeByInterface<QueryStringParameters>  => ({});

const payloadValidations =
(): YupShapeByInterface<Payload> => ({});

export default () => ({
    queryStringParameters: yup.object()
        .shape(queryStringParametersValidations()),
    payload: yup.object()
        .shape(payloadValidations()).noUnknown(),
});
    \end{lstlisting}

    \begin{lstlisting}[caption=interfaces.ts content, label={lst:interfaces_ts}]
import { RequestI } from '@restlessness/core';
export interface QueryStringParameters {}
export interface Payload {}
export interface Request extends
    RequestI<QueryStringParameters, Payload, null> {};
    \end{lstlisting}

    \subsection{Schedules}
    \label{subsec:schedules}
    Schedules are serverless functions that are triggered by a programmed event,
    such as a cron job or a rate event, an event that is fired up periodically,
    based on the time interval provided.
    By creating a Schedule from the Web Interface the framework creates the necessary
    template files under src/schedules as shown on \ref{fig:new_schedule_folder_structure},
    and also save the provided information under the config/schedules.json file.

    \begin{figure}
        \caption{Structure of a schedule endpoint folder}
        \label{fig:new_schedule_folder_structure}
        \begin{minipage}{\linewidth}
            \dirtree{%
                .1 ./.
                .2 src/.
                .3 schedules/.
                .4 clean/.
                .5 handler.ts.
                .5 index.ts.
            }
        \end{minipage}
    \end{figure}

    The structure of the template files is similar to the one generated for endpoints,
    but simpler. The handler.ts file contains the function that the developer has
    to code, while the index.ts file is the entry point.
    The core function ScheduleHandler is used to wrap the handler function, the same
    way as happens for endpoints, with the purpose of executing the framework lifecycle
    hooks.

    \begin{lstlisting}[caption=handler.ts content, label={lst:sched_handler_ts}]
export default async (event) => {};
    \end{lstlisting}

    \begin{lstlisting}[caption=index.ts content, label={lst:sched_index_ts}]
import { ScheduleHandler } from '@restlessness/core';
import handler from './handler';
export default ScheduleHandler.bind(this, handler, 'clean');
    \end{lstlisting}

    \subsection{Models}
    \label{subsec:models}
    @todo\\


    % parlare di come rln salva le info sulle varie risorse nei vari file
    % di configurazione
    % mostrare anche i files che vengono creati, tutta la struttura per gli
    % endpoints, per gli schedules etc...

    % descrivere come avviene internamente la creazione delle risorse, quindi
    % JsonConfigFile -> JsonEndpoints <-> JsonServices (principalmente sono quelle
    % le classi importanti)

    % The creation of a service is internally managed by the JsonServices class, which
    % provides the CRUD functionalities the the configuration file of the various
    % services defined. The creation of a service correspond to the creation of a new
    % configuration file under the serverless-services directory.

    % @todo talk about the 2 exporter files
    % qui magari mettere un class diagram

    \section{Test}
    A test template is also provided when creating a new endpoint (@todo ref), and
    it is based on the popular testing library \href{https://jestjs.io/}{jest}.
    % @todo talk more about the library
    In addition to the jest library, Restlessness provides also a TestHandler class,
    which makes testing the endpoint straightforward.
    Inside the beforeAll function it performs initialization operations, such as
    loading the correct environment variables, while the function invokeLambda
    executes the endpoint function providing automatically the event and context
    objects (@todo ref), simulating this way an http event.
    The fact that serverless is based on function makes possible using a simple
    testing structure as the one presented, as it's not necessary for example to
    actually starts an http server to test the endpoints.

    \begin{lstlisting}[caption=index.test.ts template, label={lst:endopints_test_ts}]
const postUsers = 'postUsers';

beforeAll(async done => {
    await TestHandler.beforeAll();
    done();
});

describe('postUsers API', () => {
    test('', async (done) => {
        const res = await TestHandler.invokeLambda(
            postUsers);
        // expect(res.statusCode).toBe(StatusCodes.OK);
        done();
    });
});

afterAll(async done => {
    await TestHandler.afterAll();
    done();
});
    \end{lstlisting}

    \section{Api documentation}
    @todo swagger generated documentation

    \section{Deploy}
    The Serverless Framework already provides a command for the deploy operation,
    as shown on \ref{sec:serverless_framework}, however with the micro services oriented
    structure suggested by Restlessness this operation is more elaborate, as it
    involves the deploy of more than one service, in a particular order.
    This is necessary because of the presence of the shared resources service, so
    to successfully deploy a service that uses resources from the shared one, it
    is necessary that those resources already exists. The correct deploy ordering
    is then shared service first, followed by all the other services.
    It should be noted that the offline service is not involved in the deploy
    process as it's used only for local development.
    To address this operations the Restlessness CLI provides a custom deploy
    command (listing \ref{lst:rln_command_deploy}), and a complementary remove command
    that removes all the services enforcing an opposite ordering.

    \begin{lstlisting}[caption=Deploy command, label={lst:rln_command_deploy}]
$ restlessness deploy
$ restlessness deploy --env production
$ restlessness deploy --env production users
    \end{lstlisting}

    It is possible to deploy the application on different environments, otherwise
    the command assume staging as the default environment.
    It is also possible to perform the deploy of just a single service, to keep
    the whole development, test and deploy process fast and easy, when making small
    changes, in accordance with the serverless paradigm.

    Since the deploy operation involves more than one service, it's important that
    the information between them are consistent, especially when deploying. This is
    why the deploy command, under the hood, takes care of performing this check,
    with a method from the JsonServices class, named healthCheck.
    In particular, it checks that the various services belong to the same serverless
    organization and organization, the same aws deploy region, and that do not exist
    services with functions associated to the same path. The latter is due to the fact
    that the services use a shared api gateway.

    \section{Environment variables}
    \label{sec:env_vars}
    An important aspect when developing web applications is the handling of different
    deploying environments, as each one of them requires different configurations,
    mostly for sensitive information, such as database credentials.
    Has been decided to handle this information with different environment files,
    storing environment variables.
    At project initialization the framework creates 4 different environments: locale,
    test, staging and production. Each environment has an associated type and
    stage (@todo ref). The type represent the purpose of that environment, below
    are the available types:
    \begin{itemize}
        \item test: environments used only for testing, which can happen locally
            but also through CI platform.
        \item dev: environments used for local development
        \item deploy: environments that can be deployed
    \end{itemize}
    All information about the environments (name, type, stage) are stored in the
    configuration file config/envs.json and are managed by the JsonEnvs class.

    Environment variables are stored in the format key=value and variable expansions
    is supported, so the value of a key can be another variable, using the syntax
    shown on listing \ref{lst:env_key_syntax}.

    \begin{lstlisting}[caption=Environment variable syntax, label={lst:env_key_syntax}]
key1=${otherKey}
key2=sample ${key1}
    \end{lstlisting}

    Each environment is then stored under the envs/ directory, in the form .env.<name>,
    and the interaction with those files is handled by the EnvFile class.
    The load and expansion operation is performed differently depending on the operation,
    local development or deploy.
    During local development it is the dev command that load the environment specified
    in input (\ref{sec:local_dev}).
    During deploy instead, the environment file is expanded and copied under the
    project root, in a file named .env, as this makes deploying from CI straightforward.
    Then at runtime the .env is automatically loaded by the LambdaHandler or ScheduleHandler
    functions (\ref{subsec:endpoints}, \ref{subsec:schedules}).

\end{chapter}
        % \begin{chapter}{Application}
    \label{chap:application}

    During its development process, the Restlessness framework has been tested on
    real deployed applications and this has been fundamental as it helped finding
    bugs and critical issues at an early stage.
    The main issues have emerged during the implementation of the backend for the
    project Spazio alla Scuola, a platform thought by the Fondazione Agnelli.

    The foundation is a non-profit, independent institute for social science research,
    born in 1966 in Turin, by the lawyer Agnelli, on the occasion of the centenary
    of the birth of the founder of Fiat, Senator Giovanni Agnelli.
    Its purpose is to work in support of scientific research and to disseminate
    knowledge of the conditions on which Italy's progress depends.

    The project Spazio alla Scuola aims to provide a concrete support to school
    leaders for lecture resumption on September 2020, given the health situation on the
    country due to the SARS-CoV-2 pandemic.
    The platform offers tools to verify the capacity of classrooms and other school spaces,
    to plan classrooms flows and staggering, in compliance with the distancing measures.
    The platform is provided as a free service and is available at the address
    \url{www.spazioallascuola.it} \cite{spazio_alla_scuola}.

    \section{Cold start}
    \label{subsec:cold_start}

    The first encountered problem has been Cold start, a new term in the serverless
    development that denotes the situation in which a serverless function is not
    active yet, so the platform must perform some resources initialization, with the
    main one being\footnote{Relatively to the Aws platform} \cite{aws_doc_runtimes}:

    \begin{itemize}
        \item Code: the project's code is uploaded in a zip archive, so it needs to
            be downloaded and extracted.
        \item Extensions: AWS allows to associate extensions to a lambda function, to
            integrate it with custom monitoring, security or other tools.
        \item Runtime: bootstrap operation for the chosen runtime environment,
            it is also possible to provide a custom runtime if needed.
        \item Function: code written by the developer, it can perform some resource
            initialization, such as creating a database connection.
    \end{itemize}

    \begin{figure}
        \centering
        \includegraphics[width=\linewidth]{source/images/aws-lambda-lifecycle.png}
        \caption{Aws Lambda lifecycle}
        \label{fig:aws_lambda_lifecycle}
    \end{figure}

    The Cold start refers exactly to this Init phase and it represents an overhead to the
    function execution. However, once this phase is completed the function is ready
    and subsequent invocations will not suffer from it. Then after some times without
    receiving any events, usually in the order of 5 to 20 minutes, the platform performs
    the Shutdown phase, so any following event causes the process to start again from
    the Init phase.
    For the majority of runtimes the duration of the Cold start varies in the order
    of tenths of a second, as shown on figure \ref{fig:cold_start_duration}. The provided
    numbers vary also based on the memory allocated for the function and the size of the
    provided code package.

    \begin{figure}
        \centering
        \includegraphics[width=\linewidth]{source/images/cold-start-duration.png}
        \caption{Cold start duration for different runtimes}
        \label{fig:cold_start_duration}
    \end{figure}

    In the particular case of the project Spazio alla Scuola the Cold start duration
    was experienced to be in the order of 1.5s, caused mainly from: mongodb
    initialization and connection (about 500ms) and third party libraries (about 400ms)
    and Restlessness overhead (about 50ms).

    One of the approaches to mitigate the effect of the Cold start proposed by the
    Serverless community has been the plugin named
    \href{https://www.npmjs.com/package/serverless-plugin-warmup}{serverless-plugin-warmup}.
    The plugin creates a scheduled function programmed to invoke the other defined functions,
    forcing the platform to keep an active container for each function.
    This way the Cold start effect remains present, but the end user of the api does
    not experience it.

    It has been decided to make this plugin an integral part of the Restlessness framework,
    granting out of the box support for it. From the Web Interface is possible to
    enabled or disable the warmup on the single endpoint, since not all functions may
    need it.
    By including the warmup plugin into the framework the effect of Cold start has
    been mitigated, however, it introduced another type of issue.

    \section{Database proxy}
    The project Spazio alla Scuola rely on the popular non relational database
    \href{https://www.mongodb.com}{mongodb}. As stated previously, each function
    run in its own runtime, independently from the others, consequently each function
    requiring database access needs to open a non shared connection.
    So the number of active connections can become quite high, depending on the number
    of active functions, furthermore, using mongodb the connection remains active for
    a certain amount of time even after the function has been shutdown.
    This leads to a high number of active connections, which is a problem, not only
    in terms of resources used, since each connection requires memory usage on the
    database, but also because mongodb has a limit of 500 concurrent connections
    and once the threshold is exceeded the application experiences random errors when
    performing database operations.
    \begin{figure}
        \centering
        \includegraphics[width=7cm]{source/images/mongo-connections.png}
        \caption{Mongodb connections reaching the 500 threshold}
    \end{figure}

    Although the problem has been amplified by the introduction of the warmup plugin
    integration, it remains a critical issue for application that rely on the high
    scalability of the serverless platform.
    To address this problem on its relational databases, AWS rely on the usage of a
    proxy between the functions and the database. Exploiting the concept of a proxy,
    it has been decided to approach the problem in the same way, since a solution for
    for the mongo database does not exist at the moment.
    Restlessness already provides the package \textit{@restlessness/dao-mongo}, as
    described on section \ref{sec:data_access_object}, defining an abstraction level
    over the mongodb driver, so it was possible to include a proxy without changing
    the exposed methods for the users.
    It has been decided to develop an open source plugin, named
    \href{https://github.com/getapper/serverless-mongo-proxy}{serverless-mongo-proxy},
    to provide the proxy functionality, independently from the Restlessness framework,
    as shown on \ref{subsec:database_proxy}.
    The dao-mongo package then uses the plugin internally, providing an effective
    solution to the presented problem.

    \section{Micro services}
    \label{subsec:application_micro_services}
    During the deployment of an application on the AWS platform a number of resources
    are created for each function, to provides services such as logging, API Gateway
    for http events, permissions and others functionalities. The AWS platform has a
    threshold of maximum 200 resources definable for each service
    (\ref{subsec:sls_disadvantage}) and since for each function there are about 10
    resources associated, it follows that each service can define about 20 functions.
    Since the serverless paradigm proposes a Micro services oriented approach this
    limitation actually force developers to compose their application as a set of
    low complexity services.
    So the next step in the Restlessness framework development has been to switch
    between the management of a single service, to a multitude of services, under
    the same Restlessness project. With this approach it has been possible to split
    the functions of the project Spazio alla Scuola into multiple services, obtaining
    a more fine grained separation between its logic components.

    \bigbreak
    In conclusion the choice of using serverless, combined with the Restlessness
    framework for the backend api of Spazio alla Scuola, brought the desired benefits
    in terms of ease of development, after the proper framework improvements described
    previously. At its peak, the api service has managed 500 thousand requests,
    demonstrating the advantage of the natural scalability of the serverless approach.

\end{chapter}

        % \begin{chapter}{Future Works}

    At the end of this development cycle, Restlessness can be defined as production
    ready, being used on real deployed app successfully.
    However its development is not completed and on its roadmap there are a series
    of features and improvements to do. While at the moment the framework supports
    only the AWS cloud provider, one of the main objective is to make the framework
    effectively platform agnostic, thus providing support for other providers, firstly
    for Google Cloud Platform and Microsoft Azure Functions. This feature represents
    a great challenge, as each provider's platform must be studied in its details
    to being able to offer the same functionalities cross platforms.

    Regarding code testing there is a structure for unit testing, but at the moment
    there is no proposed solution for integration testing. In this case, it will be
    possible to create a lightweight structure exploiting the fact that serverless
    is based on functions, as it has been done for unit testing.

    Another planned improvement is to bring all Cli functionalities on the Web
    Interface and vice versa, giving developers more flexibility when it comes
    to manage a Restlessness based project.

    Last but not least, the list of provided extensions can be increased, by
    supporting other databases or authentication methods.

\end{chapter}
        \begin{thebibliography}{9}

    \bibitem{what_is_sls_cloudflare} What is serverless
    \url{https://www.cloudflare.com/learning/serverless/what-is-serverless}

    % https://doi.org/10.1145%2F1496091.1496100    what is the cloud,
    \bibitem{cloud_computing_definition} A Break in the Clouds: Towards a Cloud
    Definition ....

    \bibitem{what_is_the_cloud} What is the cloud
    \url{https://www.cloudflare.com/learning/cloud/what-is-the-cloud/}

    \bibitem{iaas} What is IaaS
    \url{https://www.cloudflare.com/learning/cloud/what-is-iaas}

    \bibitem{paas} What is PaaS
    \url{https://www.cloudflare.com/learning/serverless/glossary/platform-as-a-service-paas}

    \bibitem{saas} What is SaaS
    \url{https://www.cloudflare.com/learning/cloud/what-is-saas}

    \bibitem{separation_of_concerns}  Dijkstra, Edsger W (1982).
    "On the role of scientific thought". Selected writings on Computing: A Personal
    Perspective. New York, NY, USA: Springer-Verlag. pp. 60–66. ISBN 0-387-90652-5.
    \url{https://www.cs.utexas.edu/users/EWD/transcriptions/EWD04xx/EWD447.html}

    \bibitem{sls_providers_comparison}
    Liang Wang, UW-Madison; Mengyuan Li and Yinqian Zhang, The Ohio State University,
    Thomas Ristenpart, Cornell Tech; Michael Swift, UW-Madison
    "Peeking Behind the Curtains of Serverless Platforms"
    ISBN 978-1-939133-02-1
    \url{https://www.usenix.org/conference/atc18/presentation/wang-liang}

    \bibitem{what_is_js} What is JavaScript
    \url{https://developer.mozilla.org/en-US/docs/Learn/JavaScript/First_steps/What_is_JavaScript}

    \bibitem{node_org} Node.js
    \url{https://nodejs.org/en}

    \bibitem{node_event_loop} Node.js Event Loop
    \url{https://nodejs.dev/learn/the-nodejs-event-loop}

    \bibitem{node_on_backend} Node.js on Backend
    \url{https://www.netguru.com/blog/node-js-backend}

    \bibitem{node_event_loop_2} Node.js Event Loop
    \url{https://www.geeksforgeeks.org/node-js-event-loop/}

    \bibitem{typescript_lang} TypeScript
    \url{https://www.typescriptlang.org/}

    \bibitem{typescript} TypeScript: What is it and when is it useful?
    \url{https://medium.com/front-end-weekly/typescript-what-is-it-when-is-it-useful-c4c41b5c4ae7}

    \bibitem{common_js} Eric Elliott (26 June 2014). Programming JavaScript Applications: Robust Web Architecture with Node, HTML5, and Modern JS Libraries. "O'Reilly Media, Inc.". pp. 87–. ISBN 978-1-4919-5027-2.
    \url{https://books.google.com/books?id=jUfnAwAAQBAJ&pg=PA87}

    \bibitem{json_iso} JSON data interchange syntax ISO
    \url{https://www.iso.org/standard/71616.html}

    \bibitem{npm_scoped_packages} Npm Scoped packages
    \url{https://docs.npmjs.com/creating-and-publishing-an-organization-scoped-package}

    \bibitem{github_issues} Mastering Issues
    \url{https://guides.github.com/features/issues}

    \bibitem{monorepo} From Monolith to Monorepo
    \url{https://medium.com/@brockreece/from-monolith-to-monorepo-19d78ffe9175}

    \bibitem{agile} Agile Manifesto
    \url{https://agilemanifesto.org/iso/it/principles.html}

    \bibitem{continuous_delivery} L. Chen, "Continuous Delivery: Huge Benefits, but Challenges Too," in IEEE Software, vol. 32, no. 2, pp. 50-54, Mar.-Apr. 2015, doi: 10.1109/MS.2015.27.
    \url{https://ieeexplore.ieee.org/document/7006384}

    % bibtex
    % @ARTICLE{7006384,  author={L. {Chen}},  journal={IEEE Software},   title={Continuous Delivery: Huge Benefits, but Challenges Too},   year={2015},  volume={32},  number={2},  pages={50-54},  doi={10.1109/MS.2015.27}}

    \bibitem{circle_ci_official} CircleCi Documentation
    \url{https://circleci.com/docs/2.0/about-circleci}

    \bibitem{aws_api_gateway_doc} API Gateway Lambda Authorizers
    \url{https://docs.aws.amazon.com/apigateway/latest/developerguide/apigateway-use-lambda-authorizer.html}

    \bibitem{react_js} React - A JavaScript library for building user interfaces
    \url{https://reactjs.org/}

    \bibitem{dom_standard} DOM Living Standard
    \url{https://dom.spec.whatwg.org/}

    \bibitem{ts_generics} Typescript - Generics
    \url{https://www.typescriptlang.org/docs/handbook/generics.html}

    \bibitem{sls_pro_cons} Serverless pros and cons
    \url{https://hackernoon.com/what-is-serverless-architecture-what-are-its-pros-and-cons-cc4b804022e9}

    \bibitem{sls_aws_doc} Serverless Framework Aws Guide
    \url{https://www.serverless.com/framework/docs/providers/aws/guide/intro}

    \bibitem{data_access_object} Data Access Object Pattern
    \url{https://www.oracle.com/java/technologies/dataaccessobject.html}

    \bibitem{spazio_alla_scuola} Spazio alla scuola
    \url{https://www.fondazioneagnelli.it/2020/07/17/spazio-alla-scuola}

    \bibitem{aws_doc_runtimes} Aws Lambda Environment
    \url{https://docs.aws.amazon.com/lambda/latest/dg/runtimes-context.html}

\end{thebibliography}

    \end{mainmatter}

\end{document}

% @TODO parlare un pò di più della dashboard -> chap 1
% @todo parlare anche di event e context, cosa contengono etc.. -> questo nel capitolo 2 quando si parla di aws!