\begin{mainmatter}

    \begin{chapter}{Introduction}
        \label{chap:intro}

        \section{Serverless}
        todo: what is it? description of the framework and description of a plain serverless
        project

        \subsection{Advantages}
        todo

        \subsection{Disadvantages}
        todo

        \section{The idea behind Restlessness}
        todo\\
        Idea behind it. General description. Which problems it aims to solve, advantages (user experience,
        scalability, coding efficiency) with respect to a plain serverless codebase,
        disadvantages. Why it is Open Source

        \section{Related Works}
        todo??
        Are there other similar framework? What are the differences? Why use restlessness instead?

        \section{Tools}
        Development tools/main software used with brief description: git/github, circle-ci, serverless framework,
        aws, jetbrain's products, slack, npm, typescript etc

    \end{chapter}


    \begin{chapter}{Restlessness}
        \label{chap:restlessness}

        Detailed description of the project's structure with description of main components/packages
        (maybe a class diagram could show easily the structure)

        micro-services structure of the project created by the framework
        handling of environment files

        \section{Project created by the framework}
        todo\\
        description of a project created usingthe framework

        \section{Core}
        todo

        \section{Cli}
        todo

        \section{Backend}
        todo

        \section{Frontend}
        todo

        \section{Testing}
        todo\\
        handling of tests for the framework itself and for the project created by the
        framework

    \end{chapter}


    \begin{chapter}{Restlessness Extensions}
        the framework has been designed with extensibility in mind

        some auth and dao extensions are already provided

        \section{Authentication}
        auth-jwt and auth-cognito packages

        \section{Database Access Object}
        dao-mongo package (plus plugin to create a database proxy)

        \subsection{Database Proxy}
        todo\\
        show data about number of connections until crash in normal
        situation and when using the proxy

    \end{chapter}


    \begin{chapter}{Development}
        \label{chap:development}

        \section{Github}
        Useful tools provided by GitHub (projects to handle tasks)
        Roadmap, development process/flow (issues, pull requests...)

        \section{Continuous Integration}
        circle-ci

    \end{chapter}


    \begin{chapter}{Application}
        \label{chap:application}

        Application to real projects with emerged problems

        \section{FGA covid school api}
        brief description of the project\\
        Problems arised by using restlessness (mainly the need for a database proxy
        and a solution for cold start, i.e. warmup plugin)


        \section{Gbsweb Claranominis api}
        brief description of the project\\
        Problems arised by using restlessness (mainly necessary migration to micro-services
        structure)

    \end{chapter}


    \begin{chapter}{Deployment}
        \label{chap:deployment}

        \section{Aws}
        todo\\
        detailed description of what resources are created when deploying

    \end{chapter}


    \begin{chapter}{Conclusions}
        todo??
    \end{chapter}

\end{mainmatter}