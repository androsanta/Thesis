\begin{chapter}{Future Works}

    At the end of this development cycle, Restlessness can be defined as production
    ready, being used on real deployed app successfully.
    However, its development is not completed and on its roadmap there are a series
    of features and improvements to do. While at the moment the framework supports
    only the AWS cloud provider, one of the main objective is to make the framework
    effectively platform agnostic, thus providing support for other providers, firstly
    for Google Cloud Platform and Microsoft Azure Functions. This feature represents
    a great challenge, as each provider's platform must be studied in its details
    to being able to offer the same functionalities cross platforms.

    Regarding code testing there is a structure for unit testing, but at the moment
    there is no proposed solution for integration testing. In this case, it will be
    possible to create a lightweight structure exploiting the fact that serverless
    is based on functions, as it has been done for unit testing.

    Another planned improvement is to bring all Cli functionalities on the Web
    Interface and vice versa, giving developers more flexibility when it comes
    to manage a Restlessness based project.

    Last but not least, the list of provided extensions can be increased, by
    supporting other databases or authentication methods.

\end{chapter}