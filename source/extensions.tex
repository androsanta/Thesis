\begin{chapter}{Restlessness Extensions}
    \label{chap:extensions}

    The framework has been designed from the beginning with the possibility of
    extending its functionalities using external packages.
    To achieve this, has been defined an AddOnPackage class, containing the
    following lifecycle hooks:
    \begin{itemize}
        \item postInstall: executed after the addon package has been installed.
            Here it's possible to perform initialization operations.
        \item postEnvCreated: executed after a new environment has been created,
            so the addon can add its own environment variables if needed.
        \item beforeEndpoint: executed before the corresponding function of an
            endpoint. Here it's possible to perform resource initialization,
            for example opening a database connection.
        \item beforeSchedule: as for endpoints, it's executed before the
            corresponding function of a schedule.
    \end{itemize}

    In addition to this class Restlessness provides also more specific classes,
    for authentication and data access.

    \section{Authentication}
    auth-jwt and auth-cognito packages

    \subsection{Usage example}
    @todo

    \section{Data Access Object}
    To simplify the creation of a Data Access Object, Restlessness provides the
    abstract class DaoPackage (listing \ref{lst:daopackage}), which extends the
    AddOnPackage class previously defined.

    \begin{lstlisting}[caption=DaoPackage class definition, label={lst:daopackage}]
abstract class DaoPackage extends AddOnPackage {
    abstract modelTemplate(modelName: string): string
}
    \end{lstlisting}

    In addition to the previously defined hooks, classes implementing DaoPackage,
    should implement also the modelTemplate method, and a base dao class, to which
    we will refer to as DaoBase. This latter class should provides the main Dao
    functionalities, while the code template returned by modelTemplate should define
    a class that extends the DaoBase one.

    Restlessness already provides a Dao package for the popular non relational
    database \href{https://www.mongodb.com/}{mongodb}, and it's available on the
    npm platform as '@restlessness/dao-mongo'.
    That package exports two main components, an implementation of the DaoPackage
    class, and a MongoBase class, the DaoBase class containing the main Dao
    functionalities for CRUD operations, as shown on listing \ref{lst:mongobase}.

    \begin{lstlisting}[caption=MongoBase class definition, label={lst:mongobase}]
export default class MongoBase {
    _id: ObjectId
    created: Date
    lastEdit: Date

    static get collectionName()
    static get dao(): MongoDao
    async getById(_id: ObjectId): Promise<boolean>
    static async getList<T>(
      query: QuerySelector<T> = {},
      limit: number = 10,
      skip: number = 0,
      sortBy: string = null,
      asc: boolean = true
    ): Promise<T[]>
    static async getCounter<T>(
        query: QuerySelector<T> = {}): Promise<number>
    async save()
    async update()
    async patch(fields: any): Promise<boolean>
    async remove<T>(): Promise<boolean>
    static async createIndex(
        keys, options): Promise<boolean>
}
    \end{lstlisting}

    Users of the package can then create models based on the MongoBase class through
    the Restlessness Web Interface (\ref{subsec:models}).
    The creation of that model is made possible by implementing the DaoPackage.modelTemplate
    method, as shown on (@todo ref model template)

    \begin{lstlisting}
const modelTemplate = (name: string): string => `
import {
    MongoBase, ObjectId
} from '@restlessness/dao-mongo';

export default class ${name} extends MongoBase {
    ['constructor']: typeof ${name}

    static get collectionName() {
        return '${pluralize(name, 2).toLowerCase()}';
    }
};
`;
    \end{lstlisting}

    \subsection{Database Proxy}
    todo\\
    show data about number of connections until crash in normal
    situation and when using the proxy

    \subsection{Usage example}
    @todo

\end{chapter}