\begin{abstract}
    The overview of services for the creation of web applications is focusing more
    and more towards a micro services oriented approach, moving away from monolithic
    structures.
    The maximum representation of this is with the serverless paradigm, which since
    2014 has seen an ever greater increase in its use and in its investments by the
    major cloud providers.
    Such a paradigm has found an implementation in the cloud model Function as a
    Service, which uses plain simple functions as its main resources.
    Serverless Framework has emerged as one of the major framework that allows the
    usage of the homonym paradigm in a simple way and it introduced a level of
    abstraction regarding the underlying structure of the chosen cloud provider.
    Despite the functionalities introduced by Serverless, the developer must take
    charge of various operations concerning indirectly the business logic of the
    application, with the main one being: to structure the code base, to define the
    various resources through the compilation of a configuration file, to define a
    unit testing structure, fundamental once the application complexity increases.
    Furthermore, based on the chosen cloud provider, the developer must find solutions
    to problems such as Cold start and limitations in resources creation.

    The Restlessness framework was born with the goal of improving the user experience
    of Serverless, providing a standard project and testing structure, a Command Line
    Interface and a local Web Interface through which is possible to completely manage
    the project and with the further goal of minimizing all operations that do not
    concern directly the application's business logic.
    The framework is provided as an Open Source package and with the possibility of
    extending its functionalities, through the use of addons, some of which are already
    present, to address common patterns, such as database access or authentication.
    During the framework development it has been possible to test it on real applications,
    thus allowing to find and correct critical issues, with the main ones being:
    Cold start handling, use of the non relational database mongodb and limitations
    on the applications structure proposed at the beginning of the framework
    development.

\end{abstract}
