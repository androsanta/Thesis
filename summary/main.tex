\documentclass{article}
\usepackage[english]{babel}
\usepackage[utf8]{inputenc}
\usepackage{graphicx}
\usepackage{caption}
\usepackage{float}
\usepackage[normalem]{ulem}
\useunder{\uline}{\ul}{}
\usepackage{rotating}
\usepackage{gensymb}
\usepackage{tikz}
\linespread{1.46}

\begin{document}

\title{Master Thesis Summary: \textit{Development, Test and Application of a framework for cloud serverless services}}

\author{Andrea Santu [s251579]}

\maketitle

\section{Introduction}
The overview of services for the creation of web applications is focusing more
and more towards a micro services oriented approach, moving away from monolithic
structures.
The maximum representation of this is with the serverless paradigm, which has
found an implementation in the cloud model Function as a Service, a model that
uses plain simple functions as its main resources.
Serverless Framework has emerged as one of the major framework that allows the
usage of the homonym paradigm in a simple way.
Despite the functionalities introduced by Serverless, the developer must take
charge of various operations concerning indirectly the business logic of the
application.
The Restlessness framework was born with the goal of improving the user experience
of Serverless, providing a standard project and testing structure, a Command Line
Interface and a local Web Interface through which is possible to completely manage
the project, and with the further goal of minimizing all operations that do not
concern directly the application's business logic.

\section{Development}
Part of the development has been spent on creating a productive development
workflow, with tools for Continuous Integration and Continuous Delivery, provided
by the CircleCi platform, and a version control system, provided by Github.
Its development has then been focused on the main components
\mbox{\textit{@restlessness/cli}} and \mbox{\textit{@restlessness/core}}, both
available on Npm, the Node.js package manager.
Users interact with the framework mainly using \mbox{\textit{@restlessness/cli}},
which provides commands for: creating a project, developing the project locally,
and deploying it on the cloud provider platform.
The Cli package depends on the \mbox{\textit{@restlessness/core}} package to provide
its functionalities, mainly resources creation and management. The core resources
that can be managed are: Endpoints and Schedules, which are functions executed in
response to http and periodic or programmed events respectively; Authorizers,
functions that perform authorization operation, granting or denying access to
functions or other resources.
The framework provides a system to extend its functionalities through addon packages.
It has been developed addon packages for common patterns, such as database access
and authentication.

\section{Application}
During the framework development, it has been possible to test it on real applications,
thus allowing to find and correct critical issues. In particular the main improvements
have been made during the development of the project \textit{Spazio alla Scuola},
with the main ones being: Cold start handling, use of the non relational database
mongodb, and limitations on the applications structure proposed at the beginning
of the framework development. Addressing those problems led the Restlessness
framework to be production ready, proving the benefits of the serverless paradigm
in conjunction with the Function as a Service model, in terms of scalability and
ease of development.

\end{document}
